\documentclass{vitae}
\usepackage{times}
\usepackage[hyphens]{url}

\renewcommand{\labelitemi}{$\diamond$}
\newcommand{\UIC}{University of Illinois at Chicago}

\author{Peter Snyder}

\address{Department of Computer Science\\
    \UIC\\
    Chicago, IL 60607 \\
\small\tt psnyde2@uic.edu}

\begin{document}

  \maketitle

  \section{Research Interests}
  \begin{description}
    \item{Computer Security, Privacy, and Cryptography.}
  \end{description}

  \section{Education}
  \begin{description}
    \item[Ph.D.~Computer Science] \hfill \textbf{2012 - Present}\\
    University of Illinois at Chicago, Chicago, IL\\
    Expected graduation date: Spring 2017

    \item[B.A.~Political Science] \hfill \textbf{2002 - 2006}~\\
    Lawrence University, Appleton, WI
  \end{description}

  \bibliographystyle{vitae}
    \nocite{cschi}
    \nocite{cscodaspy}
    \nocite{csgcasr}
    \nocite{snyder2013cloudsweeper}
  \bibliography{cv}

  \section{Research}
  \begin{description}

    \item[Measuring the Amount and Costs of Browser Complexity] ~\\
    The modern web has seen an explosion in the number of browser features implemented and available to web developers.  This research measures the increased complexity imposed by these new features, the usefulness of this added complexity to users and web authors, and the cost these new features entail to users privacy and security.

    \item[CRISP: Abstractions for Security Guarantees in Interactive Web Applications] ~\\
    Web users currently have few guarantees about the security properties of websites they visit, and the frequency and sophistication of attacks on web servers means that a site that is benign one day can become malicious the next, all invisibly to the client.  This project investigates new web systems that prioritize client security and code predictability at minimal cost to web author expressivity.

    \item[Cloudsweeper] \hfill \textbf{https://cloudsweeper.cs.uic.edu}\\
    Developed tool to measure and mitigate the frequency of plaintext password sharing in Gmail archives. The public tool allows users to redact or encrypt-in-place found passwords. The site has had over 2,500 users and has secured over 38,000 messages.
  \end{description}

  \section{Related Activities}
  \begin{description}

    \item[IGERT Fellow] \hfill \textbf{2013 - 2016}\\
    Electronic Security and Privacy IGERT Fellow

    \item[Security Advisor] \hfill \textbf{2015 - 2016}\\
    Advisor on security and privacy for citizen reporting group TIMBY.org

    \item[Invited Talk] \hfill \textbf{2015}\\
    Department of Information Engineering at the Chinese University of Hong Kong\\
    No Please, After You: Detecting Fraud in Affiliate Marketing Networks

    \item[Founder] \hfill \textbf{2015}\\
    UIC Cryptography and Privacy Reading Group

    \item[External Reviewer] \hfill \textbf{2015}\\
    Reviewed papers for IEEE Symposium on Security and Privacy 2016

    \item[External Reviewer] \hfill \textbf{2015}\\
    Reviewed papers for ACM Conference on Computer and Communications Security 2016

    \item[External Reviewer] \hfill \textbf{2014}\\
    Reviewed 3 papers for IEEE Symposium on Security and Privacy 2015

    \item[President] \hfill \textbf{2013 - 2014}\\
    UIC Computer Science Graduate Student Association

    \item[Invited Talk] \hfill \textbf{2014}\\
    No Secrets: Journalism in the Age of Surveillance\\
    Surveillance Defense: Practical Steps for Security and Privacy

    \item[Invited Talk] \hfill \textbf{2013}\\
    University of Illinois at Chicago Security Lunch\\
    Presented Dyer, Kevin P., et al. "Protocol misidentification made easy with format-transforming encryption."

    \item[Invited Talk] \hfill \textbf{2013}\\
    University of Illinois at Chicago Security Lunch\\
    Presented AlFardan, Nadhem J., and Kenneth G. Paterson. "Lucky Thirteen: Breaking the TLS and DTLS Record Protocols."

    \item[Invited Talk] \hfill \textbf{2013}\\
    University of Illinois at Chicago Advanced Programming Seminar Series\\
    Mirthful Mashups: Building Scaleable Web Applications

    \item[1st Place] \hfill \textbf{2013}\\
    Symantec Cyber Challenge Competition, a capture the flag style security competition.  Competed in Symantec's national competition.

    \item[External Reviewer] \hfill \textbf{2013}\\
    Reviewed 2 papers for Network and Distributed System Security Symposium (NDSS)

    \item[Invited Talk] \hfill \textbf{2012}\\
    University of Illinois at Chicago Advanced Programming Seminar Series\\
    Modern Web Development: From Angle Brackets to WebSockets
  \end{description}

\end{document}
