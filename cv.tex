\documentclass{vitae}
\usepackage{times}
\usepackage[hyphens]{url}

\renewcommand{\labelitemi}{$\diamond$}
\newcommand{\UIC}{University of Illinois at Chicago}

\author{Peter Snyder}

\address{Department of Computer Science\\
    \UIC\\
    Chicago, IL 60607 \\
\small\tt psnyde2@uic.edu}

\begin{document}

  \maketitle

  \section{Research Interests}
  \begin{description}
    \item{Computer Security, Privacy, and Cryptography.}
  \end{description}

  \section{Education}
  \begin{description}
    \item[Ph.D.~Computer Science] \hfill \textbf{2012 - Present}\\
    University of Illinois at Chicago, Chicago, IL\\
    Expected graduation date: Spring 2017

    \item[B.A.~Political Science] \hfill \textbf{2002 - 2006}~\\
    Lawrence University, Appleton, WI
  \end{description}

  \bibliographystyle{vitae}
    \nocite{cschi}
    \nocite{cscodaspy}
    \nocite{csgcasr}
    \nocite{snyder2013cloudsweeper}
  \bibliography{cv}

  \section{Research}
  \begin{description}
    \item[CRISP: Abstractions for Security Guarantees in Interactive Web Applications] ~\\
    Web users currently have few guarantees about the security properties of websites they visit, and the frequency and sophistication of attacks on web servers means that a site that is benign one day can become malicious the next, all invisibly to the client.  This project seeks to create higher level abstractions for creating HTML and Javascript applications, so that the client has guarantees about the security of the application being run, without the need for manual code inspection.

    \item[Cloudsweeper] \hfill \textbf{https://cloudsweeper.cs.uic.edu}\\
    Developed tool to measure and mitigate the frequency of plaintext password sharing in Gmail archives. The public tool allows users to redact or encrypt-in-place found passwords. The site has had over 2,500 users and has secured over 38,000 messages.

    \item[Mining in Mailboxes: Credentials Worms In The Email Domain] ~\\
    Measured the viability of an email credential worm by searching for passwords in a small set of seed email accounts, testing if those passwords give access to new accounts, and then repeating the attack. Simulated this attack using a university's email archives.
  \end{description}

  \section{Related Activities}
  \begin{description}
    \item[IGERT Fellow] \hfill \textbf{2013 - 2015}\\
    Electronic Security and Privacy IGERT Fellow

    \item[External Reviewer] \hfill \textbf{2014}\\
    Reivewed 3 papers for IEEE Symposium on Security and Privacy 2015

    \item[President] \hfill \textbf{2013 - 2014}\\
    UIC Computer Science Graduate Student Association

    \item[Invited Talk] \hfill \textbf{2014}\\
    No Secrets: Journalism in the Age of Surveillance\\
    Surveillance Defense: Practical Steps for Security and Privacy

    \item[Invited Talk] \hfill \textbf{2013}\\
    University of Illinois at Chicago Security Lunch\\
    Presented Dyer, Kevin P., et al. "Protocol misidentification made easy with format-transforming encryption."

    \item[Invited Talk] \hfill \textbf{2013}\\
    University of Illinois at Chicago Security Lunch\\
    Presented AlFardan, Nadhem J., and Kenneth G. Paterson. "Lucky Thirteen: Breaking the TLS and DTLS Record Protocols."

    \item[Invited Talk] \hfill \textbf{2013}\\
    University of Illinois at Chicago Advanced Programming Seminar Series\\
    Mirthful Mashups: Building Scaleable Web Applications

    \item[1st Place] \hfill \textbf{2013}\\
    Symantec Cyber Challenge Competition, a capture the flag style security competition.  Competed in Symantec's national competition.

    \item[External Reviewer] \hfill \textbf{2013}\\
    Reviewed 2 papers for Network and Distributed System Security Symposium (NDSS)

    \item[Invited Talk] \hfill \textbf{2012}\\
    University of Illinois at Chicago Advanced Programming Seminar Series\\
    Modern Web Development: From Angle Brackets to WebSockets
  \end{description}

\end{document}
