\documentclass[wideaddress]{vitae}
\usepackage{times}
\usepackage[hyphens]{url}
\usepackage{enumitem}
\usepackage{hyperref}

\urlstyle{same}
\renewcommand{\labelitemi}{$\diamond$}
\setlist[description]{leftmargin=0pt,labelindent=0pt}

\author{Peter E. Snyder}

\address{\url{https://www.peteresnyder.com} --- \href{mailto:snyderp@gmail.com}{snyderp@gmail.com}}

\begin{document}

  \maketitle

  \section{About Me}
  \begin{description}
    \item{\noindent I have over 10 years of experience improving online privacy through research,
    policy, and product development. I lead the privacy team at Brave Software,
    a privacy-focused startup with over 50 million users.
    In this role, I oversee research projects published at top privacy and security
    conferences, and the implementation of novel privacy features into Brave's Web browser.}

    \item{\noindent I also advocate for privacy in internet standards, through
    my work in the W3C and IETF. I chair privacy-related committees, conduct formal
    reviews of proposals from other browser vendors--Google, Safari, Edge--,
    and co-author new, novel privacy-improving specifications. Proposals
    I've co-authored have been implemented in popular browsers, and are already
    used by hundreds of millions of internet users.}

    \item{\noindent I am most interested privacy public policy and research
    opportunities. I am also especially intersted in positions that would allow me to
    work from Chicago.}
  \end{description}

  \section{Industry Experience}
  \begin{description}
    \item[Vice President of Privacy Engineering \& Senior Privacy Researcher] \hfill \textbf{2023 - Present}\\
    Brave Software, San Francisco, CA

    \item[Senior Director of Privacy \& Senior Privacy Researcher] \hfill \textbf{2022 - 2023}\\
    Brave Software, San Francisco, CA

    \item[Senior Privacy Researcher] \hfill \textbf{2020 - 2022}\\
    Brave Software, San Francisco, CA

    \item[Privacy Researcher] \hfill \textbf{2018 - 2020}\\
    Brave Software, San Francisco, CA
  \end{description}

  \section{Education}
  \begin{description}
    \item[Ph.D.~Computer Science] \hfill \textbf{2012 - 2018}\\
    University of Illinois at Chicago, Chicago, IL
  \end{description}

  \section{Roles and Selected Achievements}

  \subsection{Privacy in Standards Bodies}
  \begin{description}
    \item{\noindent I have contributed to and lead privacy efforts in multiple
    prominent standards bodies.}

    \item{\noindent I have co-chaired the W3C's Privacy Interest
    Group (PING) for three years, the group responsible for reviewing the privacy
    impact of proposed specifications. As a co-chair of PING, I have reviewed
    dozens of proposals and worked with (and occasionally against) browser
    developers from Google, Apple, Mozilla, and Microsoft to protect
    user privacy.}

    \item{\noindent I have also co-authored multiple privacy-improving
    proposals, including \href{https://datatracker.ietf.org/doc/draft-dss-star/}{STAR}--a
    lightweight cryptographic system for private, verifiable large scare internet measurements--,
    and \href{https://datatracker.ietf.org/doc/draft-sahib-httpbis-off-the-record/}{Request Off-the-Record}--a
    HTTP and browser enhancement to protect victims of intimate partner violence.
    Pre-standardization implementations of these proposals are currently being
    using by over 63 million internet users.}
  \end{description}

  \subsection{Privacy in Public Policy}
  \begin{description}
    \item{\noindent I have done significant work at the intersection of public
    policy and privacy engineering.}

    \item{\noindent In some cases, this has been in the development
    of technical features to help users assert legal rights provided by existing regulations.
    For example, I am a co-author of the \href{https://globalprivacycontrol.org/}{Global Privacy Control} (GPC)
    proposal, a system to allow users to easily
    assert the opt-out rights described in legislation like the California Consumer Privacy Act.
    GPC is used by over 50 million users, and is included in privacy focused
    Web browsers like Brave, DuckDuckGo, and Firefox.}

    \item{\noindent I have also been involved with policy and regulatory
    investigations of Google and Apple, and the privacy and competitiveness
    implications of their ``Privacy Sandbox'' and iOS browser restrictions.
    I've written about the negative impact these
    systems have on the openness of the Web, both for \href{https://brave.com/web-standards-at-brave/}{popular audiences},
    and for regulators in United States and UK.}
  \end{description}

  \subsection{Privacy in Product Design}
  \begin{description}
    \item{\noindent I have designed and overseen the implementation of a wide
    range of privacy features at Brave Software. I also manage the
    seven person privacy team at Brave, which handles
    the shipping and maintenance of these features for Brave's over 60 million users.}

    \item{\noindent The main focus of my work at Brave has been to synthesize and productize
    insights from industry and academic privacy research. A partial
    list of such features I developed at Brave include Brave's
    \href{https://brave.com/privacy-updates/3-fingerprint-randomization/}{unique approach to protecting users from browser fingerprinting},
    \href{https://brave.com/privacy-updates/7-ephemeral-storage/}{best-in-class protections against third-party tracking},
    \href{https://brave.com/privacy-updates/19-star/}{efficient and scalable system for product analytics}, and
    \href{https://brave.com/privacy-updates/13-pool-party-side-channels/}{defenses against colluding, malicious websites}.
    Each of these features were unique among popular browsers when I designed
    them at Brave, most remain so, and a few have even been directly adopted by larger
    browsers like Firefox and Safari.}
  \end{description}

  \bibliographystyle{vitae}
    \nocite{chouaki2023search}
    \nocite{mcquistin2023psl}
    \nocite{snyder2023poolparty}
    \nocite{randall2022uid}
    \nocite{davidson2022star}
    \nocite{smith2022blocked}
    \nocite{jueckstock2022stateful}
    \nocite{smith2021sugarcoat}
    \nocite{jueckstock2021measurements}
    \nocite{chen2021javascriptsigs}
    \nocite{snyder2020filters}
    \nocite{sjosten2020generation}
    \nocite{papadopoulos2020paywalls}
    \nocite{iqbal2020adgraph}
    \nocite{ghasemisharif2019speedreader}
    \nocite{snyder2017browser}
    \nocite{snyder2017doxing}
    \nocite{snyder2017cdf}
    \nocite{snyder2016browser}
    \nocite{snyder2016characterizing}
    \nocite{snyder2016phishing}
    \nocite{snyder2015no}
    \nocite{clark2015saw}
    \nocite{snyder2014yao}
    \nocite{snyder2014cloudsweeper}
    \nocite{snyder2013cloudsweeper}
  \bibliography{cv}

  \section{Reviewing and Academic Community Involvement}
  \subsection{Program Committee}
  \begin{tabular}{p{.05\textwidth}p{.4\textwidth}p{.05\textwidth}p{.4\textwidth}}
    \textbf{2024} & USENIX, S\&P &           \textbf{2020} & WWW, MADWeb \\
    \textbf{2023} & USENIX, S\&P, MADWeb &   \textbf{2019} & MadWeb, CSAW \\
    \textbf{2022} & USENIX, WWW, MADWeb, PEPR & \textbf{2018} & CSAW, ECRIME \\
    \textbf{2021} & USENIX, WWW, MADWeb, CCS \\
  \end{tabular}

  \subsection{External Reviewer}
  \begin{tabular}{p{.05\textwidth}p{.4\textwidth}p{.05\textwidth}p{.4\textwidth}}
    \textbf{2020} & SIGCOMM, CCR &                    \textbf{2016} & S\&P, CCS \\
    \textbf{2019} & Journal of Cybersecurity &        \textbf{2015} & CCS \\
    \textbf{2018} & CHI Late Breaking Work, SIGCOMM & \textbf{2013} & NDSS \\
    \textbf{2017} & USENIX Security, NDSS \\
  \end{tabular}

  \section{Teaching Experience}
  \begin{description}
    \item[Lead Instructor] \hfill \textbf{2017}\\
    CS342: Software Design, University of Illinois, Chicago % - \url{https://www.cs.uic.edu/~psnyder/cs342-summer2017/}

    \item[Teaching Assistant] \hfill \textbf{2017, 2015}\\
    CS450: Software Design, University of Illinois, Chicago % - \url{https://www.cs.uic.edu/~ckanich/cs450/s17/}
  \end{description}


  % \section{Other Contributions}
  % \begin{description}

  %   \item[Fingerprinting Protection Improvements in Brave Browser] \hfill \textbf{\url{https://github.com/brave/browser-laptop}}\\
  %   Improved the technique used to block fingerprinting related Web API methods
  %   to reduce the impact on non-fingerprinting related code, expanded the set
  %   of blocked Web API methods to cover additional fingerprinting methods, and
  %   worked with Brave engineers to address vulnerabilities in
  %   the browser's fingerprinting-blocking technique.

  %   \item[Web API Hardening Browser Extension] \hfill \textbf{\url{https://github.com/snyderp/web-api-manager}}\\
  %   Developed Firefox and Chrome extension to improve web privacy and security by
  %   controlling what browser functionality web sites can access. Web API
  %   access controls can be defined globally, or on a per-host level,
  %   to allow only trusted hosts access to privacy-threatening
  %   functionality, such as high resolution timers, WebGL, and WebRTC.

  %   \item[Dataset of Web API Use in Alexa 10k] \hfill \textbf{\url{http://imdc.datcat.org/collection/1-0723-8}}\\
  %   Public dataset documenting what Web API features popular sites use,
  %   both in a default web browser configuration, and with
  %   advertising and tracking blocking extensions installed.

  %   \item[CDF: Abstractions for Security Guarantees in Interactive Web Applications] \hfill \textbf{\url{https://github.com/bitslab/cdf}}\\
  %   Built client and server-side tools for implementing CDF, a document format for building dynamic, interactive web applications
  %   that provide increased security and privacy guarantees for users of commodity web browsers.

  %   \item[Cloudsweeper] \hfill \textbf{\url{https://cloudsweeper.cs.uic.edu}}\\
  %   Developed tool to measure and mitigate plaintext password sharing in Gmail archives. The
  %   tool allows users to redact or encrypt found passwords to reduce the harm
  %   of account compromise. The site has served over 2,500 users and has
  %   secured over 38,000 messages.
  % \end{description}

  % \section{Related Activities}
  % \begin{description}

  %   \item[Finalist] \hfill \textbf{2017}\\
  %   CSAW Applied Research Competition for work on browser privacy and security

  %   \item[Lead Instructor] \hfill \textbf{2017}\\
  %   CS 342: Software Design - \url{https://www.cs.uic.edu/~psnyder/cs342-summer2017/}

  %   \item[Invited Talk] \hfill \textbf{2017}\\
  %   Tandon School of Engineering, New York University\\
  %   Fifteen Minutes of Unwanted Fame: Detecting and Characterizing Doxing

  %   \item[IGERT Fellow] \hfill \textbf{2013 - 2017}\\
  %   Electronic Security and Privacy IGERT Fellow

  %   \item[Security Advisor] \hfill \textbf{2015 - 2017}\\
  %   Advisor for web and mobile application security for citizen reporting group TIMBY.org

  %   \item[President] \hfill \textbf{2013 - 2014, 2015 - 2016}\\
  %   UIC Computer Science Graduate Student Association

  %   \item[Founder] \hfill \textbf{2015 - 2016}\\
  %   UIC Cryptography and Privacy Reading Group

  %   \item[Invited Talk] \hfill \textbf{2015}\\
  %   Department of Information Engineering, Chinese University of Hong Kong\\
  %   No Please, After You: Detecting Fraud in Affiliate Marketing Networks

  %   \item[Invited Talk] \hfill \textbf{2014}\\
  %   No Secrets: Journalism in the Age of Surveillance\\
  %   Surveillance Defense: Practical Steps for Security and Privacy

  %   \item[1st Place] \hfill \textbf{2013}\\
  %   Symantec Cyber Challenge Competition, a capture the flag style security competition.
  % \end{description}
\end{document}
